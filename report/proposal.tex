\documentclass[10pt]{article}
\usepackage{graphicx}
\usepackage{geometry}
\usepackage{hyperref}
\hypersetup{
    colorlinks=true,
    linkcolor=black,
    urlcolor=blue
}
\geometry{margin=0.75in}
\setlength{\parskip}{0.3em}
\setlength{\itemsep}{0.1em}
\setlength{\parsep}{0.1em}

\title{CME307: Project proposal Transmission Expansion Planning under Load and Renewable Uncertainty}
\author{Edouard Rabasse, Siddhant Sukhani}
\date{November 2025}

\begin{document}

\maketitle

\section*{Problem Statement}
Determine optimal transmission line investments to minimize total system cost (investment + operation) while satisfying load demands and operational constraints. The model will account for uncertainty in renewable generation and load, ensuring reliability under realistic variability. Implementation using IEEE RTS-GMLC dataset.

\section*{Previous Work}
TEP studied since Garver's DC load flow (1970). MILP and decomposition methods handle large-scale systems. Recent work incorporates stochastic/robust optimization (Conejo et al., 2006; Ruiz \& Conejo, 2015) for renewable and demand uncertainty. RTS-GMLC (NREL) provides realistic benchmark with time-series data.

\section*{Data Overview}
RTS-GMLC: 3 regions, 73 buses, 120 AC lines, 1 HVDC link, 158 generators (39 CT, 31 RTPV, 25 PV, 23 Steam, 19 Hydro, 10 CC, 4 Wind, 3 Sync Cond, 1 each Storage/CSP/Nuclear/ROR), 23 storage units. Year-long time series: hourly (8,760) and 5-minute (105,120) resolution. Source files: \texttt{bus.csv}, \texttt{branch.csv}, \texttt{gen.csv}, \texttt{storage.csv}, \texttt{dc\_branch.csv}, \texttt{reserves.csv}, \texttt{timeseries\_pointers.csv}. Time series: Day-Ahead hourly load/wind/PV/hydro/CSP; Real-Time 5-minute. We will integrate hourly load profiles, wind/PV/hydro generation, and calculate load participation factors for regional-to-bus distribution.

\section*{Methodology}
\textbf{Baseline:} DC OPF with static load. Minimize operating cost subject to power balance, DC flow, generator limits, line thermal limits.

\textbf{TEP:} MILP extending DC OPF with binary investment variables for candidate lines. Big-M formulation for conditional DC flow. Objective: minimize investment + operating cost.

\textbf{Multi-Period TEP:} Two-stage: (1) Analyze time series to identify peak congestion periods (peak/avg/low or k-means), (2) Solve TEP for aggregated peak load. Integrate hourly load profiles, wind/PV/hydro variability, load participation factors.

\textbf{Solver:} Gurobi (GLPK fallback if needed)

\begin{minipage}[t]{0.48\textwidth}
\section*{Primary Metrics}
\begin{enumerate}
    \itemsep0.05em
    \item Total system cost (investment + operation)
    \item Unserved energy / Load shedding (MWh)
    \item Congestion frequency and severity
    \item Investment cost and payback period
    \item Operating cost savings vs baseline
    \item Robustness under uncertainty
    \item Period-by-period analysis
\end{enumerate}
\end{minipage}
\hfill
\begin{minipage}[t]{0.48\textwidth}
\section*{Milestones}
\textbf{Milestone A (T+7):} Load data, validate DC OPF, implement TEP MILP, integrate time series loader, calculate load participation factors, baseline results and visualizations.

\textbf{Milestone B (T+14):} Multi-period TEP with representative periods, renewable variability, load shedding option, scenario analyses, final report with build plan and sensitivity analysis.
\end{minipage}

\section*{Risks and Mitigations}
\begin{tabular}{p{0.45\linewidth} p{0.5\linewidth}}
    \textbf{Risk} & \textbf{Mitigation} \\
    \hline
    Data complexity/inconsistent units & Validate units from README, cross-check with MATPOWER, robust error checking. \\
    Solver scalability & Representative-hour clustering (peak/avg/low or k-means), limit candidates, PTDF reformulation if needed. \\
    Pyomo modeling issues & Careful parameter naming, incremental testing. \\
    Gurobi license limits & Reduce problem size, use GLPK fallback, simplified two-stage approach.
\end{tabular}

\end{document}
