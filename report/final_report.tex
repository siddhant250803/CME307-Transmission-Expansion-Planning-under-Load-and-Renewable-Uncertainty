\documentclass[10pt]{article}

\usepackage{graphicx}
\usepackage{geometry}
\usepackage{hyperref}
\usepackage{booktabs}
\usepackage{siunitx}
\usepackage{amsmath}
\usepackage{amssymb}

\hypersetup{
    colorlinks=true,
    linkcolor=black,
    urlcolor=blue,
    citecolor=black
}

\geometry{margin=0.8in}
\setlength{\parskip}{0.4em}

\sisetup{
    round-mode=places,
    round-precision=1,
    group-separator=
}

% ============================
% Biblatex configuration
% ============================
\usepackage[
    backend=biber,   % recommended with biblatex
    style=numeric,   % or authoryear, etc.
    sorting=none
]{biblatex}

\addbibresource{bibliography.bib}

\title{Transmission Expansion Planning under Load and Renewable Uncertainty}
\author{Edouard Rabasse \and Siddhant Sukhani}
\date{December 2025}

\begin{document}
\maketitle

\begin{abstract}
    We investigated transmission expansion planning (TEP) on the IEEE RTS-GMLC system~\cite{national_renewable_energy_laboratory_reliability_nodate}
    using a suite of DC optimal power flow (DC-OPF), single-period mixed-integer TEP, and time-series
    driven multi-period models inspired by classical and robust planning literature~\cite{garver1970,conejo2006,ruiz2015}.
    This report consolidates the empirical results (originally tracked in \texttt{RESULTS.md}) into a single
    narrative, highlights the quantitative findings, and documents how shortages are modeled through a
    load-shedding variant of the OPF.
\end{abstract}

\section{Data and Modeling Approach}
We rely on the RTS-GMLC static network (73 buses, 120 AC branches, 158 generators) and its hourly
time series for regional load, wind, photovoltaic (PV), and hydro production. The \texttt{src/data\_loader.py}
and \texttt{src/timeseries\_loader.py} modules align the static and dynamic data. All operational models
use Pyomo to solve DC-OPF style problems with Gurobi (and GLPK as fallback when needed). Candidate
transmission corridors are generated automatically by \texttt{src/tep.py}, and cost functions can be switched
between capacity-distance, pure distance, or capacity-only formulations to study capital expenditure assumptions.

\section{Mathematical Formulation}
Let $\mathcal{B}$ be the set of buses, $\mathcal{G}$ the set of generators, $\mathcal{L}$ the existing
ac lines, and $\mathcal{C}$ the candidate corridors. The binary variable $x_\ell$ indicates whether
candidate $\ell \in \mathcal{C}$ is built, $p_g$ is the generation of unit $g$, $\theta_b$ is the phase
angle at bus $b$, and $f_\ell$ the power flow on line or candidate $\ell$. The single-period TEP model
solves
\begin{align}
    \min \quad & \sum_{g \in \mathcal{G}} c_g p_g + \sum_{\ell \in \mathcal{C}} F_\ell x_\ell \label{eq:obj}                                                                                                                  \\
    \text{s.t.} \quad
               & \sum_{g \in \mathcal{G}_b} p_g
    + \sum_{\ell \in \mathcal{L}_b^{\text{in}}} f_\ell
    - \sum_{\ell \in \mathcal{L}_b^{\text{out}}} f_\ell
    + \sum_{\ell \in \mathcal{C}_b^{\text{in}}} f_\ell
    - \sum_{\ell \in \mathcal{C}_b^{\text{out}}} f_\ell
    = d_b      &                                                                                             & \forall b \in \mathcal{B} \label{eq:balance}                                                                   \\
               & f_\ell = B_\ell (\theta_{i(\ell)} - \theta_{j(\ell)})                                       &                                              & \forall \ell \in \mathcal{L} \cup \mathcal{C} \label{eq:dcflow} \\
               & -\bar f_\ell \le f_\ell \le \bar f_\ell                                                     &                                              & \forall \ell \in \mathcal{L} \label{eq:existinglimit}           \\
               & -\bar f_\ell x_\ell \le f_\ell \le \bar f_\ell x_\ell                                       &                                              & \forall \ell \in \mathcal{C} \label{eq:candlimit}               \\
               & \underline{p}_g \le p_g \le \bar{p}_g                                                       &                                              & \forall g \in \mathcal{G} \label{eq:genbounds}                  \\
               & \theta_{b_0} = 0                                                                            &                                              & \text{(reference bus)} \label{eq:ref}
\end{align}
where $c_g$ is the marginal production cost, $F_\ell$ the capital charge for candidate $\ell$,
$d_b$ the net load at bus $b$, $B_\ell$ the susceptance of line $\ell$, and $\bar f_\ell$
its thermal limit. Equations~\eqref{eq:balance} enforce nodal power balance, \eqref{eq:dcflow} impose DC power flow,
and \eqref{eq:candlimit} couple flows on candidate lines to their binary build decisions through a Big-M constraint.
In the multi-period implementation (\texttt{src/multi\_period\_tep.py}) the variables $p_g$, $f_\ell$, and $\theta_b$
are indexed by representative time period while the investments $x_\ell$ remain static.

To study infeasible periods or deliberate stress tests we add non-negative slack variables $s_b$
representing curtailed demand with penalty $\gamma$:
\begin{align}
    \min \quad & \sum_{g} c_g p_g + \gamma \sum_{b} s_b                                     \\
    \text{s.t.} \quad
               & \eqref{eq:balance} \text{ with } d_b \leftarrow d_b - s_b,\quad s_b \ge 0.
\end{align}
In the implementation $\gamma$ is the \texttt{shedding\_cost\_per\_mw} argument supplied to
\texttt{TEPWithLoadShedding}; we set $\gamma=\$50{,}000$/MWh in \texttt{run\_load\_shedding\_analysis.py}
    as a penalty when quantifying shortages in Section~\ref{sec:loadshed}.

    \section{Baseline DC-OPF Benchmark}
    The starting point is the static DC-OPF in \texttt{src/dc\_opf.py}. Table~\ref{tab:baseline} compares the baseline
    dispatch with the single-period TEP result (which collapses to the same dispatch because no new lines are built).
    The grid satisfies $8550$~MW of load at a \$138.97k operations cost, with three heavily loaded lines.

\begin{table}[h]
    \centering
    \begin{tabular}{lll}
        \toprule
        Metric                & {Baseline DC-OPF} & {Single-Period TEP} \\
        \midrule
        Operating cost [\$]   & 138,967           & 138,967             \\
        Total generation [MW] & 8,550             & 8,550               \\
        Total load [MW]       & 8,550             & 8,550               \\
        Investment cost [\$]  & 0                 & 0                   \\
        Congested branches    & 3                 & 3 (A27, CA-1, CB-1) \\
        Lines built           & 0                 & 0                   \\
        \bottomrule
    \end{tabular}
    \caption{Baseline performance metrics (data reproduced from \texttt{RESULTS.md}).}
    \label{tab:baseline}
\end{table}

\section{Transmission Cost Modeling and Sensitivity}
The TEP code supports three capital cost models. Table~\ref{tab:costmodels} summarizes their impact on a
representative 500~MW, 100~mile, 230~kV line (implemented in \texttt{src/example\_cost\_models.py}).
Even an aggressive \$200k/MW capacity-only assumption yields a \$100M line, while the realistic voltage-aware
distance model produces \$150M. Because the static system already redispatches around congestion, none of the cost
settings trigger investment in the single-period TEP or in the cost-sensitivity sweep (line costs between \$100k/MW
and \$1M/MW). Operating cost savings never offset the capital charges, so \textbf{no expansion was built in any scenario}.

\begin{table}[h]
    \centering
    \begin{tabular}{lll}
        \toprule
        Model                      & Formula                                              & Cost [\$] \\
        \midrule
        Capacity-distance          & \$1M/MW $\times$ 500 MW $\times$ (160.9 km / 100 km) & 804.7M    \\
        Distance-based (preferred) & \$1.5M/mile $\times$ 100 miles (230 kV)              & 150.0M    \\
        Capacity-only              & \$200k/MW $\times$ 500 MW                            & 100.0M    \\
        \bottomrule
    \end{tabular}
    \caption{Capital cost comparison for a single candidate line.}
    \label{tab:costmodels}
\end{table}

\section{Time-Series Integration and Multi-Period TEP}
To honor the promised stochastic elements, we incorporated hourly load and renewable profiles:
\begin{enumerate}
    \item \textbf{Representative periods.} \texttt{TimeseriesLoader.select\_representative\_periods} offers
          peak/average/low and $k$-means selection. For tractability we start with the simpler peak/average/low split
          and explicitly inspect periods 1--12.
    \item \textbf{Feasibility diagnostics.} Periods 1--6 are infeasible under strict P\textsubscript{min} because total
          minimum generation (3775~MW) exceeds the load (3400--3700~MW). The scripts in
          \texttt{src/analyze\_infeasibility.py} quantify the gap and suggest either relaxing minimum outputs or enabling load shedding.
    \item \textbf{Simplified multi-period TEP.} We use the two-stage approach in \texttt{src/simplified\_multi\_period\_tep.py}:
          run multiple DC-OPFs to identify congested peak periods (10--11), then solve a stress-tested TEP with a 150\%
          load multiplier, an added 500~MW industrial load at bus 122, and targeted line derates. Even under that extreme scenario
          the solver still prefers to redispatch (no new lines, zero load shedding, \$129k total cost).
\end{enumerate}

\section{Supply Shortage Stress Test}\label{sec:loadshed}
How do we represent shortages? We created
\texttt{src/run\_load\_shedding\_analysis.py}, which reuses the TEP formulation but disables candidate builds
and adds non-negative \texttt{load\_shed} variables with a \$50k/MWh penalty. For each day-ahead period we derate all
generators to 20\% of P\textsubscript{max} and scale nodal load by 40\%. The optimizer maxes out the remaining 2910~MW
of generation and sheds the balance (Table~\ref{tab:loadshed}). The resulting shortage bill---\$88--110M per period---provides
an explicit benchmark against which any resilience investment can be evaluated.

\begin{table}[h]
    \centering
    \begin{tabular}{cccccc}
        \toprule
        Period & Load [MW] & Generation [MW] & Load Shed [MW] & Load Shed [\%] & Penalty [\$M] \\
        \midrule
        1      & 4,765     & 2,910           & 1,855          & 38.9\%         & 92.8          \\
        2      & 4,669     & 2,910           & 1,759          & 37.7\%         & 88.0          \\
        3      & 4,643     & 2,910           & 1,733          & 37.3\%         & 86.7          \\
        4      & 4,659     & 2,910           & 1,749          & 37.5\%         & 87.4          \\
        5      & 4,830     & 2,910           & 1,920          & 39.8\%         & 96.0          \\
        6      & 5,108     & 2,910           & 2,198          & 43.0\%         & 109.9         \\
        \bottomrule
    \end{tabular}
    \caption{Load-shedding outcomes saved to \texttt{results/load\_shedding\_periods.csv}.}
    \label{tab:loadshed}
\end{table}

\section{Scenario-Based Robust Expansion}
To fulfill question (b) we introduced a scenario-based robust TEP model
(\texttt{src/scenario\_robust\_tep.py}) in which the binary build decisions meet multiple stress cases
simultaneously. Each scenario applies its own load factor, renewable availability, and branch derating pattern
while sharing the investment vector. Table~\ref{tab:robust-scenarios} lists the scenarios used in
\texttt{src/run\_robust\_tep.py}.

\begin{table}[h]
    \centering
    \begin{tabular}{lllll}
        \toprule
        Scenario               & Load scale & Renewable scale & Branch stress                       & Weight \\
        \midrule
        base                   & 1.00       & 1.00            & Nominal                             & 0.40   \\
        high\_load\_low\_renew & 1.20       & 0.70            & 0.85 overall, 0.40 on A27/CA-1/CB-1 & 0.35   \\
        low\_load\_high\_renew & 0.90       & 1.15            & 1.05 overall                        & 0.25   \\
        \bottomrule
    \end{tabular}
    \caption{Scenario definitions for the robust TEP experiment. Branch stress values scale the thermal limits.}
    \label{tab:robust-scenarios}
\end{table}

Solving the MILP with a reduced candidate cost of \$120k/MW yields the per-scenario operating profiles in
Table~\ref{tab:robust-results}. To remain feasible under the derated interties, the model proactively builds two
200~MW links (Bus 101 $\rightarrow$ 106 and Bus 101 $\rightarrow$ 117) at a combined \$61.3M capital charge, whereas the
single-scenario studies built nothing. This demonstrates that incorporating an explicit uncertainty set can materially
change both the investment plan and the expected operating cost.

\begin{table}[h]
    \centering
    \begin{tabular}{lccc}
        \toprule
        Scenario               & Total load [MW] & Total generation [MW] & Operating cost [\$] \\
        \midrule
        base                   & 8{,}550         & 8{,}550               & 138{,}925.58        \\
        high\_load\_low\_renew & 10{,}260        & 10{,}260              & 225{,}753.27        \\
        low\_load\_high\_renew & 7{,}695         & 7{,}695               & 129{,}078.68        \\
        \bottomrule
    \end{tabular}
    \caption{Scenario-level operating results for the robust TEP run (\texttt{results/robust\_tep\_summary.csv}).}
    \label{tab:robust-results}
\end{table}

\section{Key Findings}
\begin{itemize}
    \item \textbf{Redundancy.} The RTS-GMLC grid absorbs over 150\% of peak load plus a new industrial site
          without new transmission. Investment becomes attractive only under much cheaper line costs or significantly
          higher load growth.
    \item \textbf{Model fidelity.} The time-series integration pipeline (load participation, renewable availability limits,
          representative period selection) is in place, enabling more advanced stochastic or robust TEP formulations.
    \item \textbf{Shortage economics.} The load-shedding variant quantifies the dollar value of unmet demand,  providing a lever to compare future hardening options.
    \item \textbf{Robust planning impact.} When we require the plan to survive the explicit uncertainty set of
          Table~\ref{tab:robust-scenarios}, the solver invests \$61M in two new ties---a behavior absent from the deterministic
          runs---showing that the robustness model meaningfully alters the recommended build-out.
\end{itemize}

\printbibliography
\end{document}
